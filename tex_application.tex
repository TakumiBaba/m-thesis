\chapter{応用例}\label{chap:application}

本章では、本論文で提案されるアプリケーションを使った応用例について述べる。

\newpage

\section{同じ目的を持った人間の集まり}

例えば、イベント会場やライブ会場などは、同じ場所と状況をその場に居る人々が共有していることになる。
そのまま過ごしていれば、お互い名前も連絡先も知らない人同士ですれ違って終わってしまうが、
これが例えば見知った人ばかりならば状況に対する情報共有は、それが会話によるものであったり、
モバイルデバイスのメッセージツールであったりしながら活発になることだろう。
この2つの状況の違いをツールで埋めることで、その場に対する情報共有の勢いは最大化されることができる。

\section{偶然発生した状況に居合わせた場合}

先行研究では、情報を共有する場として設定されている場面は、あらかじめ名前の付けられたものである。
しかし例えば電車の事故などで電車内に閉じ込められた状況なども、その場の人間とは名前も連絡先も知らない人同士であり、
なおかつその場に対する情報を求めることの多い場面である。
このような突発的な人間の集合による状況の共有では、情報を管理しつつ共有するようなことは難しい。

モバイルデバイスを持っていればその場の情報にアクセス可能であるというのは、
人間の集まりを取り仕切って情報を管理する人が居ないような状況では強力であると言える。



% \section{イラクでのデモ、中国でのデモの例}
