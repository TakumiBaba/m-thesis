\chapter{考察と今後の課題、議論}
\label{chap:discussion}

本章では、本論文で提案したアプリケーションの有用性についての考察をおこなう。
本アプリケーションの実装に関する議論と考察を述べる。


\newpage

\section{プライバシーとセキュリティ}

\subsection{アクティビティのURLについて}

アクティビティに割り当てられるURLは仕様上、誰でも閲覧することが出来る。
URLの文字列は複雑なハッシュ値を持っており、
知らなければ偶然辿り着くようにはなっていない。
しかし、意思が介在すれば誰でも見られる場所にURLを共有することも可能なので、
不特定多数が閲覧するにはセンシティブな情報共有には向いていない。
そのため、投稿の揮発性を低減させた分、
投稿の内容が場所や状況に特化したものになるかどうかの議論がある。


\subsection{無線検出におけるプライバシーとセキュリティ}

Bluetoothによるデバイス検出機能を利用したサービスアプリケーションには、
プライバシーの問題が多く取り沙汰されている。
所持するデバイスを常に検出可能な状態にしておくのは、
検出されたMACアドレスとデバイスを結びつけ、本人到達性を増加させるおそれがある。
デバイスを繰り返し検出し続けることで人の位置と検出されたデバイス名から、
BluetoothデバイスのMACアドレスとデバイスの持ち主の結びつきを類推または特定することが可能である。
また、MACアドレスの持つ前半6つの文字はベンダー固有のものであり、
検出されたMACアドレスとアプリケーションで共有された情報の繋がりをユーザが閲覧できる状態にしておくと、
持っているデバイスとMACアドレスから、発言者の類推が可能になってしまう。
これも本人到達性を増加させる原因となる。
本アプリケーションの実装では、検出したMACアドレスはサーバに送信する以外の操作を行っておらず、
また、インターフェイスもMACアドレスとの結びつきを類推することがないようにしている。
今回の実装で扱った情報では問題は無いが、今後扱う情報がより多く、プライバシーに関わる内容が扱われる場合、
プライバシーに配慮した仕組みが必要になってくるだろう。


\section{Bluetoothの特性に関して}

\subsection{限られた効果範囲}

Bluetoothの10mという効果範囲は、
局所的な使用としては適しているものの、
使い始めの際には同じアプリケーションを持ったデバイスが周囲に存在しない限り、
効果を発揮することが出来ない。
ホッピングすることにより効果範囲を広げることが可能となるが、
ホッピングのための中継地点としても、ユーザが必要になってくる。
先行サービスは、人が密集している状況で示し合わせて使われるという
広まり方をしたものがあるが、そういった特殊な状況だけでなく、
普段の状況からでも利用ができるような仕組みが求められる。

\subsection{バージョン間の仕様の違い}

Bluetoothコア仕様では、バージョン4.0以降、
Bluetooth Low Energy(Bluetooth LE)などの異なる種類のBluetoothプロトコルが定義されている。
そのため今回の実装に使われているBluetooth2.1のプロトコルはクラシックBluetoothと呼ばれている。
Bluetooth LEは従来のクラシックBluetoothよりも電池消費の面で優れており、
常時使用するアプリケーションにはこちらの方が適していると言われている。
このBluetooth LEとクラシックBluetoothは検出方法の仕様に違いがあり、
今後実装の展開によっては、クラシックBluetoothの仕様を切り離すことも考えられる。
しかしその場合、クラシックBluetoothしか持たないデバイスを切り離すことにもなる。
このことと、上記に挙げたデバイス検出の問題は、考慮されるべきことだと思われる。

\subsection{リアルタイムな検出による問題}

今までの近接検出では結局、示し合わせて同じタイミングで使わなくては効果が無い
近接検出のより強力な使用法が求められる
