\chapter{評価}\label{chap:hyoka}

本章ではアプリケーション「そこにいる名無しさん」の評価について述べる。

\newpage

\section{プロトタイプへの評価}

本アプリケーションは、当初チャット機能のみを有するプロトタイプを制作しており、
つながり展、ORF2012で展示を行った。

2013年には研究室内部の友人に操作してもらい、意見を集めている。

その中で、問題と改良のきっかけとなった代表的な意見を以下に挙げる。

\subsection{アプリケーション自体の話題拡散性について}

意見では、「このアプリケーション自体はバズらないのではないか」
バズるとは、ここではSNSやニュースサイトなどのインターネットのシーンで話題になることを言う。

プロトタイプのような、情報共有される範囲を空間位置的に制限した上に、
その場でコミュニケーションが完結するシステムでは、例え局所的に使われることがあっても
SNSやニュースサイトなどの公共のシーンで気づかれることはない。
そのため、アプリケーションが有用であっても、広まるのは非常に困難だろうという指摘があった。

改良版では情報共有を行った後の閲覧と編集に重点を置き、情報共有のためのWebインターフェイスを追加した。

\subsection{発言内容の出現・消滅タイミングによる本人到達性について}

プロトタイプでは、検出したデバイスの判定から逐次、アクティビティ画面を完全に更新していた。
それにより、現在どの発言者が近くにいて、どの発言者が離れたか推察できてしまう可能性の問題があった。

そのため、アクティビティ画面では投稿の順序や距離による順序を気にしないよう、
ユーザごとに独自に積み重なってゆくインターフェイスへ改良した。
