% ■ アブストラクトの出力 ■
%	◆書式:
%		begin{jabstract}〜end{jabstract}	:日本語のアブストラクト
%		begin{eabstract}〜end{eabstract}	:英語のアブストラクト
%		※ 不要ならばコマンドごと消せば出力されない。



% 日本語のアブストラクト
\begin{jabstract}

モバイルデバイスの進化につれて、
一般的なモバイルデバイスに共通して搭載されているセンサやモジュールは豊富になった。
同時に、それらと組み合わせることで実現されるContext-basedなサービスも一般化している。
これらのサービスはコンテキストに特化した情報の取得を促進させつつも、
情報はサービス内限定の閉鎖的な文脈となることが多い。

対して、近年大きく発展してきたSNSやコミュニケーションサービスは、
Webの持つ自由度を持ったまま進化を続けてきた。
しかし、同じ場所に居合わせた人同士がこのようなサービスの上で情報をやりとりするには
あらかじめサービス上で連絡先を教え合わなくてはならない。
そのため、見知らぬ人同士で互いのデジタルな情報まで辿り着くには、壁があった。

本論文では、
お互いに近接した人同士が状況や場所に関する情報を取得し、共有するためのシステムを提案する。

本システムは、ユーザの近接情報を元にした情報取得と共有を促進させ、
さらにWeb上での閲覧と編集を可能にすることで、コンテキストに対するWebからの情報取得を可能にする。
Context-basedなサービスとWebサービスの特性を組み合わせることで、
近接した人同士が互いに情報をやりとりする行為を簡便なものへと発展させつつ、
Context-basedなサービスの閉鎖性を緩和させる。

本論文では、提案するシステムの実装と試用の際の議論から、システムの有効性について考察し、
近接関係に基づく情報取得の展望を示す。


\end{jabstract}



% 英語のアブストラクト
\begin{eabstract}

Through the evolution of mobile devices,
Sensors and modules that are mounted on the general mobile devices was abundant.
At the same time, the Context-based service that is realized by combining them has also been generalized.


\end{eabstract}
